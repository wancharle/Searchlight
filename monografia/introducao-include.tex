\section{Contextualização}
Crowdsourcing é a prática de obtenção de serviços necessários, idéias ou conteúdo solicitando contribuições de um grande grupo de pessoas e, especialmente, a partir de uma comunidade on-line, ao invés de funcionários ou fornecedores tradicionais \footnote{\label{wiki-crowd} \url{http://en.wikipedia.org/wiki/Crowdsourcing}}.
Existem diversos tipos de crowdsourcing, mas nesse trabalho iremos nos focar no crowdsoucing conhecido como \emph{Wisdom of the Crowd} que é um tipo de crowdsourcing que coleta grandes quantidades de informação e as agrega para obter uma imagem completa e precisa sobre um determinado tema. Nesse tipo de crowdsourcing se, por exemplo, a taxa de criminalidade de uma cidade for considerada como tema, podemos obter uma imagem completa e precisa sobre o tema através de mapas de crowdsourcing, que exibam claramente as zonas com maior e menor indices de incidentes na cidade.

A produção de mapas crowdsourcing é geralmente feita de forma automática, usualmente temos algum software e/ou site que coleta as informações e as agrupa por meio de algum algorítimo desenvolvido especificamente para o mapa.

A coleta dessas informações pode ocorrer de várias formas, tanto manual como automática. Por exemplo, no site \url{http://portoalegre.cc} os usuários podem adicionar novas informações através do site, ou seja, ela é feita de forma manual. Alguns sistemas podem coletar informações de forma automática usando programas de computadores, aplicativos de smartphones \cite{thiagarajan_cooperative_2010}  e até mesmo a própria internet, como exemplo informações do twitter\footnote{Trends do twitter em tempo real: \url{http://trendsmap.com}}.

Uma vez coletada, essa informção é analisada e exibida em um mapa de crowdsourcing, mas as vezes o mapa em si não é o produto final e sim algumas informações específicas retiradas dele. Por exemplo, podemos ter mapas que mostrem os congestionamentos no trânsito de uma cidade e um sistema \cite{thiagarajan_vtrack:_2009} que através desse mapa consegue identificar  uma rota mais eficiente com menor consumo de energia, evitando assim ficar parado no congestionamento gastando gasolina.

Mapas de crowdsourcing geralmente possuem informações posicionadas em regiões muito próximas entre si, que devido a quantidade elevada, acabam poluindo a visualização e dificultando a compreensão do mapa. Esse problema pode ser resolvido quando os mapas oferecem mecanismos para agrupar e filtrar essas informações. Um mecanismo ideal é um zoom contextual ou zoom em grupo, que filtre informações irrelevantes, em determinados níveis de zoom, deixando o mapa mais leve e compreensível.

O Projeto Searchlight pretende ser uma ferramenta para auxiliar e melhorar a visualização de mapas de crowdsourcing.

O escopo do projeto atinge a criação de uma ferramenta que visualize mapas de crowdsourcing em um navegador de internet, tanto desktop quanto mobile, usando recursos de visualização de mapas já disponíveis em HTML5 mas que ainda não possuem zoom contextual e outras opções úteis que permitam uma melhor visualização do mapa.


\section{Definição do Problema}

Mapas de crowdsourcing, por lidar com informações coletadas de um grande número de pessoas,  tendem a mostrar uma enorme quantidade de informação. 
Essa caracteristica faz com que a visualização e a compreensão do mapa seja comprometida.
 
Ao trabalhar com mapas de crowdsourcing, geralmente encontramos 2 grandes problemas: a sobreposição de informações e o zoom arbitrário. 

\subsection{Sobreposição de Informações}

\subsection{Zoom arbitrário}

\section{Objetivos}

\section{Contribuições}

\section{Estrutura da Monografia}
