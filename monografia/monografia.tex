% ------------------------------------------------------------------------
% ------------------------------------------------------------------------
% abnTeX2: Modelo de Trabalho Acadêmico (tese de doutorado, dissertação de
% mestrado e trabalhos monográficos em geral) em conformidade com 
% ABNT NBR 14724:2011: Informação e documentação - Trabalhos acadêmicos -
% Apresentação
% ------------------------------------------------------------------------
% ------------------------------------------------------------------------

\documentclass[12pt,oneside,openany,a4paper]{abntex2}	% frente e verso
%\documentclass[12pt,oneside,a4paper]{abntex2}			% apenas frente

% ---
% Pacotes fundamentais 
% ---
\usepackage{cmap}				% Mapear caracteres especiais no PDF
\usepackage{lmodern}			% Usa a fonte Latin Modern			
\usepackage[T1]{fontenc}		% Seleção de códigos de fonte.
\usepackage[utf8]{inputenc}		% Determina a codificação utiizada (conversão automática dos acentos)
\usepackage{makeidx}            % Cria o indice
\usepackage{hyperref}  			% Controla a formação do índice
\usepackage{lastpage}			% Usado pela Ficha catalográfica
\usepackage{indentfirst}		% Indenta o primeiro parágrafo de cada seção.
\usepackage{nomencl} 			% Lista de simbolos
\usepackage{color}				% Controle das cores
\usepackage{graphicx}			% Inclusão de gráficos

\usepackage[portuguese]{algorithm2e}

% ---
	
% ---
% Pacotes adicionais, usados apenas no âmbito do Modelo Canônico do abnteX2
% ---
\usepackage{lipsum}				% para geração de dummy text
% ---

% ---
% Pacotes de citações
% ---
\usepackage[brazilian,hyperpageref]{backref}	 % Paginas com as citações na bibl
\usepackage[alf]{abntex2cite}	% Citações padrão ABNT
% --- 
% CONFIGURAÇÕES DE PACOTES
% --- 
\graphicspath{{imagens/}} 
% ---
% Configurações do pacote backref
% Usado sem a opção hyperpageref de backref
\renewcommand{\backrefpagesname}{Citado na(s) página(s):~}
% Texto padrão antes do número das páginas
\renewcommand{\backref}{}
% Define os textos da citação
\renewcommand*{\backrefalt}[4]{
	\ifcase #1 %
		Nenhuma citação no texto.%
	\or
		Citado na página #2.%
	\else
		Citado #1 vezes nas páginas #2.%
	\fi}%
% ---

% ---
% Informações de dados para CAPA e FOLHA DE ROSTO
% ---
\titulo{SEARCHLIGHT: Melhorando a visualização de informações crowdsourcing em Mapas Web através de Zoom seletivo}
\autor{Wancharle Sebastião Quirino}
\local{Vitória - ES, Brasil}
\data{7 de Junho de 2013}
\orientador{Celso Alberto Saibel Santos}
\instituicao{%
  Universidade Federal do Espírito Santo
  \par
  Centro Tecnológico
  \par
  Departamento de Informática}
\tipotrabalho{Trabalho de Conclusão de Curso}
% O preambulo deve conter o tipo do trabalho, o objetivo, 
% o nome da instituição e a área de concentração 
\preambulo{Monografia apresentada para obtenção do Grau de Bacharel em Engenharia de Computação pela Universidade Federal do Espírito Santo.}
% ---



% ---
% Configurações de aparência do PDF final

% alterando o aspecto da cor azul
\definecolor{blue}{RGB}{41,5,195}

% informações do PDF
\hypersetup{
     	%backref=true,
     	%pagebackref=true,
		%bookmarks=true,         		% show bookmarks bar?
		pdftitle={\imprimirtitulo}, 
		pdfauthor={\imprimirautor},
    	pdfsubject={\imprimirpreambulo},
		pdfkeywords={PALAVRAS}{CHAVES}{abnt}{abntex}{abntex2},
	    pdfproducer={LaTeX with abnTeX2}, 	% producer of the document
	    pdfcreator={\imprimirautor},
    	colorlinks=true,       		% false: boxed links; true: colored links
    	linkcolor=blue,          	% color of internal links
    	citecolor=blue,        		% color of links to bibliography
    	filecolor=magenta,      		% color of file links
		urlcolor=blue,
		bookmarksdepth=4
}
% --- 

% --- 
% Espaçamentos entre linhas e parágrafos 
% --- 

% O tamanho do parágrafo é dado por:
\setlength{\parindent}{1.3cm}

% Controle do espaçamento entre um parágrafo e outro:
\setlength{\parskip}{0.2cm}  % tente também \onelineskip

% Controles do espaçamento entre linhas:
%\OnehalfSpacing	% espaçamento um e meio (padrão); 
%\DoubleSpacing		% espaçamento duplo
%\SingleSpacing		% espaçamento simples	
% --- 
	

% ---
% compila o indice
% ---
\makeindex
% ---
\makenomenclature
% ---

% ----
% Início do documento
% ----
\begin{document}

% ----------------------------------------------------------
% ELEMENTOS PRÉ-TEXTUAIS
% ----------------------------------------------------------
% \pretextual

% ---
% Capa
% ---
\imprimircapa
% ---

% ---
% Folha de rosto
% (o * indica que haverá a ficha bibliográfica)
% ---
\imprimirfolhaderosto*
% ---

% ---
% Inserir a ficha bibliografica
% ---

% Isto é um exemplo de Ficha Catalográfica, ou ``Dados internacionais de
% catalogação-na-publicação''. Você pode utilizar este modelo como referência. 
% Porém, provavelmente a biblioteca da sua universidade lhe fornecerá um PDF
% com a ficha catalográfica definitiva após a defesa do trabalho. Quando estiver
% com o documento, salve-o como PDF no diretório do seu projeto e substitua todo
% o conteúdo de implementação deste arquivo pelo comando abaixo:
%
% \begin{fichacatalografica}
%     \includepdf{fig_ficha_catalografica.pdf}
% \end{fichacatalografica}
% ---


% ---
% Inserir folha de aprovação
% ---

% Isto é um exemplo de Folha de aprovação, elemento obrigatório da NBR
% 14724/2011 (seção 4.2.1.3). Você pode utilizar este modelo até a aprovação
% do trabalho. Após isso, substitua todo o conteúdo deste arquivo por uma
% imagem da página assinada pela banca com o comando abaixo:
%
% \includepdf{folhadeaprovacao_final.pdf}
%
\begin{folhadeaprovacao}

  \begin{center}
    \vspace*{1cm}
    {\ABNTEXchapterfont\large\imprimirautor}

    \vspace*{\fill}\vspace*{\fill}
    {\ABNTEXchapterfont\bfseries\Large\imprimirtitulo}
    \vspace*{\fill}
    
    \hspace{.45\textwidth}
    \begin{minipage}{.5\textwidth}
        \imprimirpreambulo
    \end{minipage}%
    \vspace*{\fill}
   \end{center}
    
   Trabalho aprovado. \imprimirlocal, 7 de junho de 2013:

   \assinatura{\textbf{\imprimirorientador} \\ Orientador} 
   \assinatura{\textbf{José Gonçalves Pereira Filho} \\ Convidado 1}
   \assinatura{\textbf{Magnos Martinello} \\ Convidado 2}
   %\assinatura{\textbf{Professor} \\ Convidado 3}
   %\assinatura{\textbf{Professor} \\ Convidado 4}
      
   \begin{center}
    \vspace*{0.5cm}
    {\large\imprimirlocal}
    \par
    {\large\imprimirdata}
    \vspace*{1cm}
  \end{center}
  
\end{folhadeaprovacao}
% ---

% ---
% Dedicatória
% ---
\begin{dedicatoria}
   \vspace*{\fill}
   \noindent
   \begin{flushright}
   
   \textit{ Este trabalho é dedicado aos meus pais,  que sempre estiveram comigo, \\ por toda a sua dedicação.}
   \end{flushright}
   
\end{dedicatoria}
% ---

% ---
% Agradecimentos
% ---
%\begin{agradecimentos}
%Os agradecimentos principais são direcionados a  ...

%Agradecimentos especiais são direcionados a ...

%\end{agradecimentos}
% ---

% ---
% Epígrafe
% ---

% ---

% ---
% RESUMOS
% ---

% resumo em português
\begin{resumo}
Crowdsourcing é  a prática de obtenção de serviços, ideias ou conteúdo solicitando contribuições de um grande grupo de pessoas. Mapas de crowdsourcing tendem a mostrar uma enorme quantidade de informação. Essa característica faz com que a visualização e  compreensão do mapa seja comprometida.
Neste contexto, o presente trabalho utiliza algumas estratégias para lidar com o excesso de marcadores em Mapas Web. 

O objetivo deste trabalho é melhorar a visualização das informações de crowdsourcing em mapas Web, por meio da criação de um framework que aplique essas estratégias e permita um controle seletivo do zoom.  Além disso, o framework deve melhorar o acesso a essas informações fornecendo um mecanismo de automatizar a criação e compartilhamento desse tipo de mapa.
Sendo assim, o trabalho contribui para automatizar o processo de criação de mapas web com informações de crowdsourcing e para melhorar a visualização e compreensão dos mesmos.

% objetivo, o método, os resultados e as conclusões do documento. A ordem e a extensão
% destes itens dependem do tipo de resumo (informativo ou indicativo) e do
% tratamento que cada item recebe no documento original. O resumo deve ser
% precedido da referência do documento, com exceção do resumo inserido no
% próprio documento. (\ldots) As palavras-chave devem figurar logo abaixo do
% resumo, antecedidas da expressão Palavras-chave:, separadas entre si por
% ponto e finalizadas também por ponto.

 \vspace{\onelineskip}
    
 \noindent
\textbf{Palavras-chaves}: Crowdsourcing, Mapas Web,  Visualização de Dados, Zoom, Agrupamento.
\end{resumo}

% resumo em inglês
\begin{resumo}[Abstract]
Crowdsourcing is the practice of obtaining services, ideas or content so by requesting contributions from a large group of people. Crowdsourcing Maps tend to show a huge amount of information. This feature makes the viewing and understanding of the map compromised. In this context, the present work uses some strategies to deal with the extra markers in Web Maps.

The objective of this work is to improve the visualization of crowdsourcing information on Web Maps, by creating a framework to implement these strategies and allows a selective control of the zoom. Furthermore, the framework should improve access to this information by providing a mechanism for automate the creation and sharing of this type of map. Thus, the work helps to automate the process of creating Web Maps with crowdsourcing information and  to improve the visualization and understanding of them.

\vspace{\onelineskip}
 
\noindent 
\textbf{Key-words}: Crowdsoucing, Web Maps, Data visualization, Zoom, Clustering.
\end{resumo}

% ---

% ---
% inserir lista de ilustrações
% ---
\pdfbookmark[0]{\listfigurename}{lof}
\listoffigures*
\cleardoublepage
% ---

% ---
% inserir lista de tabelas
% ---
%\pdfbookmark[0]{\listtablename}{lot}
%\listoftables*
%\cleardoublepage
%% ---

% ---
% inserir lista de abreviaturas e siglas
% A lista de Abreviaturas e Siglas pode ser facilmente montada com o pacote 
% nomencl. Abaixo segue um exemplo.
% ---
%\nomenclature{Fig.}{Figura}
%\nomenclature{$A_i$}{Area of the $i^{th}$ component} 
%\nomenclature{456}{Isto é um número}
%\nomenclature{123}{Isto é outro número}
%\nomenclature{a}{primeira letra do alfabeto}
%\nomenclature{lauro}{este é meu nome} 

%\renewcommand{\nomname}{Lista de abreviaturas e siglas}
%\pdfbookmark[0]{\nomname}{las}
%\printnomenclature
%\cleardoublepage
% ---

% ---
% inserir lista de símbolos
% ---
% O abnTeX2 não provê mecanismo para lista de símbolos.
% ---

% ---
% inserir o sumario
% ---
\pdfbookmark[0]{\contentsname}{toc}
\tableofcontents*
\cleardoublepage
% ---



% ----------------------------------------------------------
% ELEMENTOS TEXTUAIS
% ----------------------------------------------------------
% É possível usar \textual ou \mainmatter, que é a macro padrão do memoir.  
\mainmatter

% ----------------------------------------------------------
% Introdução
% ----------------------------------------------------------

\section{Contextualização}
Crowdsourcing é a prática de obtenção de serviços necessários, idéias ou conteúdo solicitando contribuições de um grande grupo de pessoas e, especialmente, a partir de uma comunidade on-line, ao invés de funcionários ou fornecedores tradicionais \footnote{\label{wiki-crowd} Definição completa de crowdsourcing \url{ http://en.wikipedia.org/wiki/Crowdsourcing}}.
Existem diversos tipos de crowdsourcing, mas nesse trabalho iremos nos focar no crowdsourcing conhecido como \emph{Wisdom of the Crowd} que é um tipo de crowdsourcing que coleta grandes quantidades de informação e as agrega para obter uma visão completa e precisa sobre um determinado tema. Essa visão, dependo do tema de estudo, pode ser representada por um mapa de crowdsourcing.

A produção de mapas crowdsourcing é geralmente feita de forma automática, usualmente temos algum software e/ou site que coleta as informações e as agrupa por meio de algum algorítimo desenvolvido especificamente para o mapa.

A coleta dessas informações pode ocorrer de várias formas, tanto manual como automática. Por exemplo, no site \citeauthoronline{portoalegre} os usuários podem adicionar novas informações através do site, ou seja, ela é feita de forma manual. Alguns sistemas podem coletar informações de forma automática usando programas de computadores, aplicativos de smartphones \cite{thiagarajan_cooperative_2010}  e até mesmo da internet\footnote{ Mapa com informações coletadas do Twitter \url{http://trendsmap.com}}.

Uma vez coletada, essa informação é analisada e exibida em um mapa, mas as vezes o mapa em si não é o produto final e sim algumas informações específicas retiradas dele. Por exemplo, podemos ter mapas que mostrem os congestionamentos no trânsito de uma cidade e um sistema \cite{thiagarajan_vtrack:_2009} que através desse mapa consegue identificar  uma rota mais eficiente com menor consumo de energia, evitando assim ficar parado no congestionamento gastando gasolina.

Em alguns casos, mapas de crowdsourcing possuem informações posicionadas em regiões muito próximas entre si, que devido a quantidade elevada, acabam poluindo a visualização e dificultando a compreensão do mapa. Esse problema pode ser resolvido quando os mapas oferecem mecanismos para agrupar e filtrar essas informações. Um mecanismo ideal é o zoom contextual ou zoom em grupo, que filtra informações irrelevantes, em determinados níveis de zoom, deixando o mapa mais leve e compreensível.

O Projeto Searchlight pretende ser uma ferramenta para auxiliar e melhorar a visualização de mapas de crowdsourcing.

O escopo do projeto atinge a criação de uma ferramenta que visualize mapas de crowdsourcing em um navegador de internet, tanto desktop quanto mobile, usando recursos de visualização de mapas já disponíveis em HTML5 mas que ainda não possuem zoom contextual e outras opções úteis que permitam uma melhor visualização do mapa.



\section{Definição do Problema}
Mapas de crowdsourcing tendem a mostrar uma enorme quantidade de informação. Essa característica faz com que, em alguns casos, a visualização e a compreensão do mapa seja comprometida.
 
Ao trabalhar com mapas de crowdsourcing, geralmente encontramos 2 problemas: a sobreposição de informações e o zoom arbitrário. 



\subsection{Sobreposição de Informações}
O site \citeauthoronline{portoalegre} é um exemplo da importância do mapas de crowdsourcing no contexto governamental e na sociedade. Por meio desse site os cidadãos de porto alegre podem relatar os problemas de sua cidade para que as autoridades tomem as devidas providências. 

Um dos principais objetivos do site é identificar as áreas prioritárias em que o governo deveria atuar. Mas a sobreposição de informações dificulta essa tarefa, pois ocorre frequentemente nesse site. 

\begin{figure}[htb]
	\caption{\label{fig-porto-alegre} Sobreposição de informações no mapa do site PortoAlegre.cc}
	\begin{center}
	    \includegraphics[scale=0.4]{portoalegre-cc}
	\end{center}
	\legend{Fonte: \citeonline{portoalegre}}
\end{figure}

Esse problema fica evidente na \autoref{fig-porto-alegre} quando consideramos, a possibilidade, que um grupo de 5 marcadores reunidos numa região específica, podem sobrepor dezenas ou até milhares de outros marcadores. Ou seja, o mapa não consegue mostrar, com clareza e precisão, as áreas de maior ocorrência de determinado incidente. 

O site fornece um filtro por categorias, que diminui de forma significativa a quantidade de informação exibida.  Mas infelizmente não resolve o problema, pois a sobreposição de informação ainda pode ocorrer com informações de uma mesma categoria.



\subsection{Zoom arbitrário}
Alguns sites criam mecanismos que minimizam o problema da sobreposição de informações. Como exemplo, temos o \citeonline{crimemapatl}  mostrado na \autoref{fig-mapatl} que mostra um mapa com a taxa de crimes em Atlanta. 

O site fornece filtros por categoria, data e zoom. Mas o principal responsável pela eliminação da sobreposição de informação é o filtro por zoom. Esse filtro agrupa todos os marcadores que estão sobrepostos, no zoom atual, em um único marcador que exibe a informação somada dos marcadores que o compõem.
 
\begin{figure}[htb]
	\caption{\label{fig-mapatl} MapATL não possui sobreposição de informação devido ao filtro por zoom}
	\begin{center}
	    \includegraphics[scale=0.4]{crime-mapatl-com-filtro}
	\end{center}
	\legend{Fonte: \citeonline{crimemapatl}}
\end{figure}

Segundo \cite[42,44]{silva2010solap+} esse tipo de agrupamento, baseado em grelha, pode ser implementado a partir do algorítimo WaveCluster\cite{wavecluster}. 

 

\begin{figure}[htb]
	\caption{\label{fig-zoomab} Entre ZOOM A e ZOOM C existem muitos níveis intermediários e não apenas 1 (ZOOM B).}
	\begin{center}
	    \includegraphics[scale=0.6]{zoomab}
	\end{center}
	\legend{Fonte: \citeonline{crimemapatl}}
\end{figure}

Esse algorítimo resolve o problema de sobreposição de informações. Porém o problema do zoom arbitrário permanece, ilustrado na \autoref{fig-zoomab}, permanece.

O usuário precisa aplicar vários zoons para ir do ZOOM A para o ZOOM C. Mas durante essa interação é gasto tempo e banda, da conexão deinternet, do usuário para exibir os zoons intermediários, quando o ideal seria exibir apenas o ZOOM B.

Esse gasto de banda, prejudica\footnote{dispositivos móveis que usam mapas vetoriais, como o iphone da apple, não sofrem desse problema.} a usabilidade de mapas em dispositivos móveis pois, geralmente, eles possuem pouca banda de internet.

Uma abordagem para esse problema é o uso de um zoom inteligente que siga uma hierarquia espacial invés de de simplesmente dobrar a visualização espacial. 
 





\section{Objetivos}

\section{Contribuições}

\section{Estrutura da Monografia}



\section{Fundamentos de Mapas Geográficos}
	\subsection{Coordenadas E Zoom} usar texto da figura do livro do google
	\subsection{Marcadores e Icones} pag 140
	\subsection{InfoWindow} videos e fotos pagina 152
	\subsection{Polylines}

\section{Estratégias para lidar com muitos marcadores}
  \subsection{Reduzir numero de marcadores}
	\subsubsection{Pesquisa}
	\subsubsection{Filtro}
	\subsubsection{Otimizar a representação}
nem sempre usar marcadores para representar cada parte de um elemento... linhas nao precisam de marcadores para cada vertice , e grupos proximos podem de ser representados por unico poligono.
 pagina 198
 
 (pagina 88 ,silva, tabela 6, figura 93, figura 65 ())

  \subsection{Agrupamento/Clustering}
  pagina 199 (google)
		\subsubsection{Por grelha}
		\subsubsection{Por distancia}
		\subsubsection{Por região}
		
\section{Algorítimos de agrupamento}
	\subsection{Métodos baseados em grelha}
		\subsubsection{WaveCluster}
	\subsection{Aplicações}
		\subsubsection{MarkerCluster}
		pagina 206 a 212
		\subsubsection{MarkerClustered}



\section{Planilhas eletrônicas e Mapas}
\subsection{O desafio chinês}
mostra o uso de planilhas pelo governo chines \cite{chinaPlanilha}
\subsection{Domínios de conhecimento}
Mostra a importância do uso de planilhas \cite{credinePlanilha} 
\subsection{Usando planilhas como fonte de dados para Mapas Geográficos}
Mostra \cite{lieberman2009spatio}

% -------------------------------------
Neste capítulo será detalhado o processo de desenvolvimento do framework Searchlight, e as ferramentas utilizadas para a construção do mesmo.

\section{Considerações Iniciais}
Durante  a fase de pesquisa deste projeto, descobriu-se uma variedade de recursos que poderiam ser usados na elaboração de um framework para visualização de mapas. Muitos destes recursos já estavam no cronograma do projeto, outros foram adicionados  e alguns ficaram fora do cronograma e colocados na lista de trabalhos futuros.

Inicialmente a meta do framework era apenas encontrar uma maneira de visualizar mapas de crowdsourcing de forma mais limpa e sucinta através de técnicas de agrupamento aliadas a hierarquia de zoom. Mas no decorrer das pesquisas percebeu-se que seria interessante adicionar alguns recursos como filtro por categoria e foco em grupo. 

Também foi adicionado ao projeto a possibilidade de gerar e compartilhar um mapa web sem precisar escrever uma linha de código.
O único requisito é fornecer um endereço web onde o gerador de mapas deve buscar os dados geográficos do mapa. Neste caso, o usuário não precisa mais saber programar para poder usufruir dos recursos do framework e a única preocupação dele fica depositada sobre o conteúdo do mapa.

Em relação ao conteúdo usado pelo compartilhador de mapas, o framework atualmente suporta o armazenamento numa planilha eletrônica ou em um arquivo de formato JSON\footnote{Formato muito usado para troca de dados entre websites \url{http://www.json.org}}. Quando o conteúdo é oriundo de planilha eletrônica é necessário que esta esteja armazenada no Google Docs e seja pública. De forma similar, quando o conteúdo é armazenado em um arquivo JSON é necessário que o website que hospeda o arquivo suporte o protocolo JSONP. Caso contrário, o usuário deve instalar o framework Searchlight no website que deseja visualizar o mapa.


Em relação a API de mapas utilizada pelo projeto, o objetivo inicial era usar a api do Google Maps para a visualização dos mapas. Porém,  na época em que esta pesquisa era feita, circulava notícias que o Google estaria mudando seus termos de serviço e iria começar a cobrar\footnote{Uma notícia sobre o inicio das cobranças do Google Maps \url{http://goo.gl/f1E3K}} pelo uso da API do Google Maps. Além disso, no mesmo período, a Apple anunciou que iria abandonar o uso do Google Maps em seus smartphones e adotaria uma solução própria. Isto trouxe a necessidade de procurar uma alternativa ao Google Maps. A alternativa encontrada foi a biblioteca open-source Leaflet.jss \cite{leaflet} que será descrita no decorrer deste capítulo.

\subsection{Agrupamento de Pontos}
		Durante a fase de pesquisa, deste trabalho, foi feito uma busca por ferramentas acadêmicas, e comerciais, que já fizessem o uso de algorítimos de agrupamento de pontos. Foram encontradas duas ferramentas promissoras: MarkerClusterer e Leaflet.MarkerCluster.
		
		
		\subsubsection{MarkerClusterer}		
		MarkerClusterer \cite[188]{livroGoogleApiV3} é uma ferramenta fornecida pelo google que estende o uso da API do Google Maps para o uso de algorítimos de agrupamentos.
		
		MarkerClusterer é uma solução baseada em grelha. Ela agrupa marcadores de acordo com sua distância ao centro de um grupo. Quando um marcador é adicionado, ele pesquisa sua posição em todos os grupos. Caso não seja colocado em nenhum grupo, um novo grupo é criado para este marcador. A \autoref{fig-markerclusterer} demonstra a aplicação da ferramenta nos marcadores de um mapa.
	\begin{figure}[htb]
	\caption{\label{fig-markerclusterer}Exemplo de utilização da API MarkerClusterer }
	\begin{center}
	    \includegraphics[scale=0.8]{markerclusterer}
	\end{center}
		\legend{Fonte: \cite[figura 20]{silva2010solap+}  }
\end{figure}

		\subsubsection{Leaflet.MarkerCluster}
		Assim como a biblioteca MarkerCLusterer estende a API do Google Maps para uso de agrupamento de marcadores, o plugin Leaflet.MarkerCluster\cite{gitleafletmarker} serve como uma extensão para a API Leaflet. 
		
		Este plugin, faz basicamente as mesmas funções da ferramenta MarkerCluster, porém com algumas melhorias. Por exemplo, ao fazer zoom o mapa exibe uma pequena animação do processo de agrupamento; ao se passar o mouse sobre um marcador do grupo o mapa exibe um polígono que mostra os limites alcançados pelo grupo; os grupos que não são visíveis  na visualização atual do mapa são retirados do mapa para aumentar a performance. Além dessas melhorias, uma que merece destaque é a sua incrível capacidade de customização. 

		Ao contrário da biblioteca MarkerClusterer, o plugin Leaflet.MarkerCluster possui um documentação bem completa que pode ser encontrada em \cite{gitleafletmarker}. 
		
		É importante observar, que inicialmente, procurou-se por uma documentação mais abrangente da biblioteca MarkerCluster do Google Maps, mas até o momento da confecção desse trabalho não foi possível encontrar nada além de alguns poucos exemplos de uso. Isto, e outros fatores, contribuíram para a escolha da biblioteca Leaflet.js como API de visualização de mapas do framework Searchlight.
		  

\section{Ferramentas Utilizadas}

Um framework, em desenvolvimento de software, é uma abstração que une códigos comuns entre vários projetos de software provendo uma funcionalidade genérica. Um framework pode atingir uma funcionalidade específica, por configuração, durante a programação de uma aplicação. Ao contrário das bibliotecas, é o framework quem dita o fluxo de controle da aplicação, chamado de Inversão de Controle\footnote{Definição completa em: \url{http://pt.wikipedia.org/wiki/Framework}}.

De modo geral, um framework é conjunto de ferramentas que trabalham em conjunto. O que une as ferramentas são as regras do framework que tem como objetivo prover as funcionalidades que motivaram a criação do framework.

O framework Searchlight tem como objetivo facilitar a criação e a visualização de mapas de crowdsourcing por meio de automatização do processo de criação do mapa. O público alvo abrange  programadores e usuários sem nenhum conhecimento de programação. Para atingir este objetivo o framework faz uso do seguinte conjunto de ferramentas: GitHub, Lealflet.js,  Tabletop.js e RapydScript. 



\subsection{Github}
GitHub é um Serviço de Web Hosting Compartilhado para projetos que usam o controle de versionamento Git. Mais do que isso, Github incorpora recursos sociais e de pesquisa que na prática o transformam em uma rede social para programadores. GitHub incorpora elementos tanto do facebook como do twitter. Por exemplo é possivel favoritar um projeto, seguir um programador, ver quais sãos as tendencias de programação, quais projetos estão se destacando, quais projetos se relacionam entre si e muitos outros recursos.

Neste projeto o github foi usado para hospedar o código fonte da solução desenvolvida e o código latex desta monografia. Todos esses dados podem ser acessados através do endereço \url{https://github.com/wancharle/Searchlight}. 

Na \autoref{fig-gitsource} podemos observar a pagina do código fonte do projeto. A partir dela um programador pode reportar um bug, solicitar permissões para contribuir com o projeto, copiar o código fonte e muitos outros recursos. 

	\begin{figure}[htb]
	\caption{\label{fig-gitsource}Pagina do github.com que hospeda o código fonte deste projeto.}
	\begin{center}
	    \includegraphics[scale=0.5]{gitsource}
	\end{center}
\end{figure}

Além de hospedar os códigos fontes do projeto, o GitHub também participa ativamente do framework Searchlight. Seu papel no framework é realizado através da ferramenta GitHub Pages. Essa ferramenta estende o uso do GitHub para que ele se comporte como um servidor de paginas web simples. Isto permite ao framework Searchlight hospedar a pagina de geração e compartilhamento de mapas sem que seja preciso contratar um servidor web apenas para esse propósito.


\subsection{Leaflet.js}

Leaflet é uma biblioteca open-source feita em JavaScript para visualização de mapas interativos. Seu design é baseado em simplicidade, performance e usabilidade. Devido a isso é bastante amigável a dispositivos móveis como tablets e smartphones.

Leaflet é uma biblioteca moderna e robusta, que tira vantagem dos recursos de navegadores modernos como o uso de HTML5  e CSS3 mas ainda mantém suporte aos navegadores antigos.  É bastante confiável e sua estabilidade já foi testada por grandes sites da internet como Foursquare\footnote{Foursquare abandona Google Maps em prol de leaflet.js \url{http://goo.gl/JXPCW}} e Flickr.

Neste projeto, a biblioteca Leaflet é utilizada para fornecer os recursos de exibição e interação com mapas na web.  Além disso, o framework a extensão  \textbf{Leaflet.Spin}, para controlar o processo de carregamento de dados, e Leaflet.MarkerCluster, para agrupar marcadores.


\subsection{TableTop.js}

Tabletop\cite{tabletop} é uma biblioteca JavaScript que se comunica com o Google Docs e permite a leitura de planilhas eletrônicas por websites via Javascript. 

O framework Searchlight utiliza a biblioteca Tabletop na geração e compartilhamento de mapas. A função da biblioteca neste caso é ler as planilhas com as informações do mapa e converter os dados para o formato JSON que é um dos formatos de dados mais utilizado em JavaScript.


\subsection{Rapydscript}

Rapydscript \cite{rapydscript} é um pré-compilador para JavaScript com sintaxe bem próxima da linguagem Python. Sua principal vantagem, é a facilidade de se programar orientado a objetos de forma bem similar como é feito Python. 

A maior parte do framework Searchlight é escrita usando a sintaxe do Rapydscript, as partes restantes são escritas em JavaScript nativo.

Decidiu-se usar um spré-compilador para JavaScript pois a linguagem JavaScript, apesar de ser uma linguagem com suporte a orientação a objetos, não possui uma sintaxe agradável para criação de classes de objetos. Devido a isso, é possível encontrar na internet algumas dezenas de pré-compiladores de javascript\footnote{O site \url{http://altjs.org} exibe uma lista dos principais pré-compiladores de JavaScript.}
. 

\section{Desenvolvimento do Sistema}
\subsection{Requisitos do Sistema}
\subsection{Arquitetura do Sistema}
	\subsubsection{Processamento de dados}
	\subsubsection{Compartilhamento de mapas}


\section{Site do projeto}
\section{Recursos desenvolvidos}
\subsection{Filtro por categorias}
\subsubsection{Filtro por data}
\subsubsection{Campo de pesquisa}

\subsection{Agrupamento de marcadores}
\subsection{Foco em Grupo de marcadores}

\subsection{Geração automática de mapas}
\subsubsection{Marcadores com link,imagens e videos}
\subsection{Compartilhamento de mapas}



% ---
% Conclusão
% ---
%\chapter*[Conclusão]{Conclusão}
%\addcontentsline{toc}{chapter}{Conclusão}

\section{trabalhos futuros}
	Otimizações visuais (polylines, grupos, graficos, numericos), melhoria em icones, leitura de mais fontes de dados.
	\subsection{Redes de Grupo}
	usar polylines para representar redes de grupo
\section{Considerações finais}


% ---
% Finaliza a parte no bookmark do PDF, para que se inicie o bookmark na raiz
% ---
\bookmarksetup{startatroot}% 
% ---






% ----------------------------------------------------------
% ELEMENTOS PÓS-TEXTUAIS
% ----------------------------------------------------------
\postextual


% ----------------------------------------------------------
% Referências bibliográficas
% ----------------------------------------------------------
\bibliography{monografia}

% ----------------------------------------------------------
% Glossário
% ----------------------------------------------------------
%
% Consulte o manual da classe abntex2 para orientações sobre o glossário.
%
%\glossary

% ----------------------------------------------------------
% Apêndices
% ----------------------------------------------------------

%% ---
%% Inicia os apêndices
%% ---
%\begin{apendicesenv}
%
%% Imprime uma página indicando o início dos apêndices
%\partapendices
%
%
%\end{apendicesenv}
%% ---
%
%
%% ----------------------------------------------------------
%% Anexos
%% ----------------------------------------------------------
%
%% ---
%% Inicia os anexos
%% ---
%\begin{anexosenv}
%
%% Imprime uma página indicando o início dos anexos
%\partanexos
%
%
%\end{anexosenv}

%---------------------------------------------------------------------
% INDICE REMISSIVO
%---------------------------------------------------------------------

% \cleardoublepage
% \phantomsection 
\printindex

\end{document}
